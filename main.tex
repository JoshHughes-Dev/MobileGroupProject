\documentclass{article}
\usepackage[utf8]{inputenc}
\usepackage{fancyvrb,fullpage,framed,graphicx}
\def\code#1{\texttt{#1}} % use \code{} to format text as programming code

\graphicspath{ {c:/users/coas7/desktop/ } } % will need to be changed to correct path

\title{15COC155 Mobile Application Development: \\
    Student Organiser app}
\author{Group 8 - Dan Cohen, Joshua Hughes, Niall Rose \& Alex Smith}
\date{May 2016}

\begin{document}

\maketitle

\tableofcontents
\newpage

\section{Introduction}
We were asked to propose, specify and develop a native Android application (app) fulfilling certain technical requirements. The app we developed helps students maintain engagement with their course at Loughborough University by integrating with their university calendar and tracking their progress on attendance.

The app largely meets the initial mandatory and optional specifications. This report summarises the application, describes the specifics of its design with some explanation into the methodology used and discusses some of the app's limitations.

\section{Specification}
Maintaining ongoing motivation and reliably attending contact sessions are challenges experienced by many university students on long full-time courses. Since it has repeatedly been shown that there is strong positive correlation between attendance and performance, it can be confidently said that waning motivation in students constitutes for them a real, tangible problem (in addition to the obviously negative, but nonetheless subjective impact on their mood). 

The scalable, inherently mathematical nature of technology makes it an ideal tool with which to solve the kind of organisation problem previously outlined. Furthermore, most students are young and many have access to personal technology devices such as mobile phones and tablets with which they already manage their personal and social commitments. 

It would seem then that there exists space for a technological solution to the problem, both in terms of the appropriateness of technology here and in terms of the students' ability to incorporate such a tool into their pre-existing routine.

The proposed application will therefore be a university ``personal assistant / organiser" mobile application on the Android platform for university students (specifically of Loughborough University), designed to help students track and encourage their own attendance to lectures/seminars/labs as well as to help them organise other events such as group assignment meetings and deadlines. The application differs from a simple calendar application in that it focuses heavily on tracking one's own progress, providing myriad feedback through dialogues and animations designed to stimulate and encourage engagement.

The application's objective is provide students the power to address their own lecture attendance and organise their day-to-day activities and commitments. The application is primarily designed from the student's perspective as a self management tool rather than to assist the university's own internal attendance monitoring, although wide adoption would undoubtedly provide indirect benefit to the university as well.

The following features are likely to be included in some form:

\begin{itemize}
    \item Integration of the user's Loughborough University (Google) account - supporting use of Android's built-in account manager with which the user will doubtless be familiar.
    \item Synchronisation with Google Calendar - users may export and import events to and from their existing university calendar.
    \item Locally stored information supporting geolocation tracking - e.g. names and geolocation coordinates of campus buildings.
    \item Push notification reminders of upcoming events using location data - e.g. ``Your upcoming [lecture] is [2 miles] away and starts in [20 minutes]!''.
    \item Automatic attendance registration through two-step geolocation tracking - the user's location can be tracked twice during a session's time slot, minimising the ability of the user to deceive the application into falsely registering attendance. 
    \item Responsible geolocation tracking - tracking will be carried out only when strictly necessary in order to preserve device resources.
    \item Interfacing with external applications - e.g. a ``get directions" action may prompt the Google Maps application to carry out an operation.
    \item Full compliance with Google material design - the application will sit flush alongside other applications and present an intuitive and familiar interface to the user.
    \item Intuitive touch gestures - e.g. double tap to create a new entry, long press and drag to move events, etc.
    \item Sharing support - a share button can allow the user to share new events with other users who may be interested.
    \item Frequent, informative feedback - features such as weekly and daily statistics and graphical weekly summaries will allow users to track and visualise their progress. Notifications help users to keep engaged, e.g. ``You didn't attend this lecture last week, try attending it this week."
    \item Gamification of progress feedback - presenting features of the application as a game in order to enhance user feeling of reward.
\end{itemize}

Some preliminary interface designs are included in a separate accompanying document, outlining some of the planned features and how they will fit to the device screen width. Diagrams \ref{d1}, \ref{d2} and \ref{d3} below provide an indication into the planned direction of the user interface (UI).
\newline
\noindent\emph{***UI original diagrams here***}

\section{Implementation}
This section describes the app in terms of its fulfilment of the initial requirements set out by the lecturer. The app largely meets the requirements while providing useful function and in that sense can be argued to be a success. Students are able to benefit from the app's statistics logging, convenient navigation service (in the case of particular buildings on the campus) and helpful sharing tools provided that they have added their university calendar to their Android phone. Despite these positives, the app is is somewhat limited - both in part by our technical implementation and in part by traits inherent in its design. 

\subsection{Structural overview}
``An overview of the structure of the software at both application and programming level...''

\subsection{Limitations}
The app is unable  ...

\subsection{Fulfilment of the Initial Requirements}
This section describes the technical fulfilment of the initial requirements along with screenshots to demonstrate.

\subsubsection{Distinct UI Screens}
Many distinctly different screens are used to display information appropriately, e.g. the day view, the 3-day view, the week view and the event details view. The following screenshots \ref{uiDay}, \ref{uiThreeDay}, \ref{uiWeek} and \ref{uiEventDetails} respectively show these different screens.

\emph{***UI diagrams here***}

\subsubsection{Compliance with the Activity Lifecycle}
The various application components are used appropriately and are able to communicate with each other, resulting in a stable app that can be easily managed by the OS - in particular each Activity and Service are declared and used appropriately. \emph{*** further detail about classes when finished*** }
\subsubsection{Use of Permissions}
The app requires the following permissions:\\
    \newline
    \indent \code{android.permission.INTERNET}\\
    \indent \code{android.permission.ACCESS\_NETWORK\_STATE}\\
    \indent \code{android.permission.GET\_ACCOUNTS}\\
    \indent \code{android.permission.READ\_CALENDAR}\\
    \indent \code{android.permission.WRITE\_CALENDAR}\\
    \indent \emph{... ***other permissions*** ...}\\
    \newline
Only those permissions that are strictly required by the app are requested. In this way the app can claim to be responsible with its use of permissions, an important trait in the current age where privacy is of increasing concern in the minds of many people. 

In particular, the app abides by Google's requirements for use of permissions in version 6.0 Marshmallow (the most recent public release at time of writing) by making requests on demand rather than on load.

\subsubsection{Use of Intents}
The app makes use of two external Intents, the first allows the user to share their personal commentary on a calendar event with any other app that supports text input for the use case scenario of sharing comments or notes with fellow students on the same course as the user, while the second allows the user to view the building location on a maps application, e.g. Google Maps, including directions from their current location.
\subsubsection{Custom ContentProvider}
The app makes use of a custom Calendar Provider, comprising the classes \code{$\mathtt{\sim}$/data/Calendar}, \code{$\mathtt{\sim}$/data/Event} and \code{$\mathtt{\sim}$/data/CalendarProvider}.
\subsubsection{Custom Loader}
The app makes use of a custom Loader in tandem with its custom Calendar Provider in order to load the calendar event data. It should be noted that this is not an ideal use for an asynchronous Loader given that the app requires all of the event data to be present in order to function correctly, however it is a suitable way to demonstrate the use of a custom Loader given the app's scope and the limited timeframe for development.
\subsubsection{Use of Local storage}
Local storage is used to store some of the calendar data in the form of a SQLite database.
\subsubsection{Custom View}
The app makes use of a custom View...
\subsubsection{Use of ShareActionProvider}
The app makes use of Android's ShareActionProvider to allow users to share their personal commentary on events that they have in common. In this way users are able to append additional information to a shared event, something they would not be able to do otherwise (events managed by the university's shared calendars cannot be edited by users).
\subsubsection{Use of Services}
The app makes use of several Android services including LocationManager to provide users with directions to buildings on the university campus... 
\subsubsection{Use of Notifications}
When the device is not online, the app notifies the user that the latest calendar data may not be being used. In practice this is unlikely to represent a significant issue for users (how often does the university really update its shared calendars?) but it is still appropriate for the user to be informed.
\subsubsection{Use of Touch gestures}
A swipe gesture is utilised to access the navigation menu drawer from the left of the screen.

\section{Testing}
Some testing...

\section{Conclusion}
``A discussion on challenges and limitations...''\\
\newline
Challenges:
\begin{itemize}
    \item Marshmallow permissions (brand new, small demographic, poor documentations)
    \item Custom content provider documentation poor due to inherently wide scope 
    \item View challenges (primarily tablet view because of rotations)
\end{itemize}
Limitations
\begin{itemize}
    \item geolocation on/off? accuracy? (wifi/gps?) signing operates over very small spaces
    \item too much UI given that all events take place between 8 and 6
    \item only have the signing data to work with (limited scope for generating statistics)
    \item dependency on the university a) managing their calendar correctly and b) maintaining the same format etc.
    \item Loughborough only 
    \item timezones??
\end{itemize}
\noindent\textbf{$Top$ $quality$ $product$ $m8$.}

\end{document}
